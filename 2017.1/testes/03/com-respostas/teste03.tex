\documentclass[12pt,a4paper,oneside]{article}

\usepackage[utf8]{inputenc}
\usepackage[portuguese]{babel}
\usepackage[T1]{fontenc}
\usepackage{amsmath}
\usepackage{amsfonts}
\usepackage{amssymb}
\usepackage{graphicx}

\usepackage{xcolor}
% Definindo novas cores
\definecolor{verde}{rgb}{0.25,0.5,0.35}
\definecolor{jpurple}{rgb}{0.5,0,0.35}
% Configurando layout para mostrar codigos Java
\usepackage{listings}
\lstset{
  language=Java,
  basicstyle=\ttfamily\small, 
  keywordstyle=\color{jpurple}\bfseries,
  stringstyle=\color{red},
  commentstyle=\color{verde},
  morecomment=[s][\color{blue}]{/**}{*/},
  extendedchars=true, 
  showspaces=false, 
  showstringspaces=false, 
  numbers=left,
  numberstyle=\tiny,
  breaklines=true, 
  backgroundcolor=\color{cyan!10}, 
  breakautoindent=true, 
  captionpos=b,
  xleftmargin=0pt,
  tabsize=4
}

\author{\\Universidade Federal de Goiás (UFG) - Regional Jataí\\Bacharelado em Ciência da Computação \\Teoria de Grafos \\Esdras Lins Bispo Jr.}

\title{\sc \huge Terceiro Teste}

\date{11 de julho de 2017}

\begin{document}

\maketitle

{\bf ORIENTAÇÕES PARA A RESOLUÇÃO}

\footnotesize

\begin{itemize}
	\item A avaliação é individual, sem consulta;
	\item A pontuação máxima desta avaliação é 10,0 (dez) pontos, sendo uma das 06 (seis) componentes que formarão a média final da disciplina: quatro testes, uma prova e os exercícios de aquecimento;
	\item A média final ($MF$) será calculada assim como se segue
	\begin{eqnarray}
		MF & = & MIN(10, S) \nonumber \\
		S & = & (\sum_{i=1}^{4} 0,2.T_i ) + 0,2.P  + 0,1.EA \nonumber
	\end{eqnarray}
	em que 
	\begin{itemize}
		\item $S$ é o somatório da pontuação de todas as avaliações,
		\item $T_i$ é a pontuação obtida no teste $i$,
		\item $P$ é a pontuação obtida na prova, e
		\item $EA$ é a pontuação total dos exercícios de aquecimento.
	\end{itemize}
	\item O conteúdo exigido compreende os seguintes pontos apresentados no Plano de Ensino da disciplina: (3) Subgrafos, (4) Grafos Conexos e Componentes, (5) Cortes e Pontes, e (10) Outros tópicos.
\end{itemize}


\begin{center}
	\fbox{\large Nome: \hspace{10cm}}
\end{center}

\newpage

\normalsize

\begin{enumerate}

	\item (5,0 pt) {\bf [E 1.144]} Sejam $G$ e $H$ dois grafos conexos tais que $V_G \cap V_H \not= \emptyset$. Mostre que o grafo $G \cup H$ é conexo. 
	
	
	{\color{blue} {\bf Resposta:}
		Sejam $x$ e $y$ dois vértices quaisquer de $G \cup H$. Se existir um caminho que liga $x$ a $y$, $G \cup H$ é conexo. Isto é verdade, pois:
		\begin{itemize}
			\item \underline{se $x, y \in V_G$}, então existe um caminho que liga $x$ a $y$ (pois $G$ é conexo);
			\item \underline{se $x, y \in V_H$}, então existe um caminho que liga $x$ a $y$ (pois $H$ é conexo); e
			\item \underline{se $x \in V_G$ e $y \in V_H$ (ou $x \in V_H$ e $y \in V_G$)}, então existe um caminho que liga $x$ a $z$, e existe um caminho que liga $z$ a $y$ (sendo $z$ um dos vértices de $V_G \cap V_H$). Assim, existe um caminho que liga $x$ a $y$. 
		\end{itemize}
	
		Logo, $G \cup H$ é conexo $\blacksquare$
	} \newpage
	
	\item (5,0 pt) Na linguagem C, complete as lacunas da função {\tt ehCircuito} conforme apresentada abaixo. Você deve substituir apenas as linhas 6, 12 e 16, pelos trechos de código 1, 2 e 3, respectivamente. O objetivo é que esta função retorne valor 1, se o grafo fornecido como parâmetro for um circuito. O valor de retorno será 0, caso não for.

\begin{lstlisting}[language=C]
int ehCircuito(Grafo *g){

	if( g->n < 3)
		return 0;
	if( g->n == 3)
		// TRECHO 1
	else{
		int vGrau2 = -1;
		int qtdGrau2 = 0;
		for(int v=0; v<g->n; v++){
			if( grau(g,v) == 2){
				// TRECHO 2
			}
		}
	
		//TRECHO 3
		
		Grafo *h = gMenosV(g, vGrau2);
		return ehCaminho(h);
	}
}\end{lstlisting}

{\color{blue} \bf Resposta: } \\
\begin{lstlisting}[language=C]
//TRECHO 1
for(int v=0; v<g->n; v++)
	if(grau(g,v) != 2)
		return 0;
return 1;
\end{lstlisting}

\begin{lstlisting}[language=C]
//TRECHO 2
vGrau2 = v;
qtdGrau2++;
\end{lstlisting}

\begin{lstlisting}[language=C]
//TRECHO 3
if( vGrau2 == -1 || qtdGrau2 != g->n)
	return 0;
\end{lstlisting}
	
\end{enumerate}

\newpage

\section{grafos.h}

\begin{lstlisting}[language=C]
#ifndef GRAFO_H_INCLUDED
#define GRAFO_H_INCLUDED

#include <stdlib.h>

typedef struct grafo {
	char *nome;
	int n;       // Numero de vertices
	int m;       // Numero de arestas
	int **adj;   // Ponteiro para a matriz de adjacencias
} Grafo;

int **gerarMatriz(int n) {

	int i, j;
	
	// Aloca um vetor de n ponteiros (do tipo int) na memoria
	int **m = malloc( n * sizeof (int *));
	
	// Para cada ponteiro, aloca n elementos (do tipo int) na memoria
	for (i = 0; i < n; i++)
		m[i] = malloc( n * sizeof (int));
	
	// Preenche cada celula da matriz com valor 0
	for (i = 0; i < n; i++)
		for (j = 0; j < n; j++)
			m[i][j] = 0;
	
	return m;
}

void inicializar(Grafo *g, char *nome, int n) {

	// Inicializa os parametros da estrutura Grafo
	g->nome = nome;
	g->n = n;
	g->m = 0;
	g->adj = gerarMatriz(n);

}

void inserirAresta(Grafo *g, int v, int w) {

	// Garantindo que os argumentos estao entre
	// 0 e n-1 (fechado)
	if( (v >= 0) && (v < g->n) ){
		if( (w >= 0) && (w < g->n) ){
		
			if (g->adj[v][w] == 0) { // Verifica se a aresta nao existe
				// E necessario preencher as duas celulas com 1
				g->adj[v][w] = 1;
				g->adj[w][v] = 1;
				g->m++;
			}
		}
	}

}

void removerAresta(Grafo *g, int v, int w) {

// Garantindo que os argumentos estao entre
	// 0 e n-1 (fechado)
	if( (v >= 0) && (v < g->n) ){
		if( (w >= 0) && (w < g->n) ){
		
			if (g->adj[v][w] == 1) { // Verifica se a aresta existe
				// Eh necessario preencher as duas celulas com 0
				g->adj[v][w] = 0;
				g->adj[w][v] = 0;
				g->m--;
			}
		}
	}

}


void exibir(Grafo *g) {

	int v, w;
	
	printf("\n=================");
	printf("\n%s", g->nome);
	printf("\n=================");
	
	// Imprime o conjunto de vertices
	printf("\nV = {");
	
	for(v = 0; v < g->n; v++ )
		printf("%d, ", v);
	
	if(g->n > 0)
		printf("\b\b");
	printf("}");


	// Imprime o conjunto de arestas
	printf("\nA = {");
	for (v = 0; v < g->n; v++)
		for (w = 0; w < g->n; w++)
			if (g->adj[v][w] == 1) 
				if (v < w)
					printf("(%d, %d); ", v, w); 
	
	if(g->m > 0)
		printf("\b\b");
	printf("}");
	
	printf("\n=================\n");
}

int grau(Grafo *g, int v){

	int soma;
	
	if( (v >= 0) && (v < g->n) ){
		soma = 0;
		for(int i=0; i<g->n; i++)
		soma += g->adj[v][i];
	}

	return soma;
}

int grauMinimo(Grafo *g){

	int minimo = g->n;
	
	for(int i=0; i < g->n; i++){
	
		int grauI = grau(g, i);
		
		if( grauI < minimo )
			minimo = grauI;
	}
	
	return minimo;
}

int grauMaximo(Grafo *g){

	int maximo = -1;
	
	for(int i=0; i < g->n; i++){
	
		int grauI = grau(g, i);
		
		if( grauI > maximo )
			maximo = grauI;
	}
	
	return maximo;
}

int ehRegular(Grafo *g){

	if(g->n != 1){
		int valorGrau = grau(g, 0);
		
		for(int i=1; i<g->n; i++)
			if( grau(g,i) != valorGrau )
				return 0;
	}
	
	return 1;
}

int eh3Regular(Grafo *g){

	if(g->n != 1){
		int valorGrau = grau(g, 0);
	
		if(valorGrau != 3)
			return 0;
		
		for(int i=1; i<g->n; i++)
			if( grau(g,i) != valorGrau )
				return 0;
	
		return 1;
	}
	
	return 0;
}

Grafo *gMenosV(Grafo *g, int v){

	Grafo *h = malloc(sizeof (Grafo));
	inicializar(h, "G - v", g->n - 1);
	
	for(int i=0; i<g->n; i++){
	
		int l = i;
		
		if(i == v) continue;
		else
		if(i > v)
			l = i - 1;
		
		for(int j=0; j<g->n; j++){
		
			int c = j;
			
			if(j == v) continue;
			else
			if(j > v)
				c = j - 1;
			
			h->adj[l][c] = g->adj[i][j];
		}
	}

	return h;
}

int ehCaminho(Grafo *g){

	if( g->n == 1)
		return 1;
	else{
		//Encontrar um vertice de grau 1 e
		//contar a qtd de vertices de grau 1.
		int vGrau1 = -1;
		int qtdGrau1 = 0;
		for(int v=0; v<g->n; v++){
			if( grau(g,v) == 1){
				vGrau1 = v;
				qtdGrau1++;
			}
		}
	
		//Se nao houver vertice de grau 1 ou
		//se nao houver dois vertices de grau 1, nao eh caminho
		if( vGrau1 == -1 || qtdGrau1 != 2)
			return 0;
		
		//Criar H = G - v (vertice de grau 1)
		Grafo *h = gMenosV(g, vGrau1);
		
		//Chamada recursiva para H
		return ehCaminho(h);
	}
}

#endif // GRAFO_H_INCLUDED\end{lstlisting}

\newpage

\section{fabrica.h}

\begin{lstlisting}[language=C]
#ifndef FABRICA_H_INCLUDED
#define FABRICA_H_INCLUDED

#include <stdlib.h>
#include "grafo.h"

Grafo *k(int n){

	Grafo *g = malloc(sizeof (Grafo));
	
	inicializar(g, "K_x", n);
	
	for(int i=0; i<n; i++)
		for(int j=0; j<n; j++)
			if(i != j)
				inserirAresta(g, i, j);
	
	return g;
}

Grafo *caminho(int n){  // Comprimento n-1

	Grafo *g = malloc(sizeof (Grafo));
	
	inicializar(g, "Caminho (Comp. x)", n);
	
	for(int i=0; i < n-1; i++)
		inserirAresta(g, i, i+1);
	
	return g;
}

Grafo *circuito(int n){  // Comprimento n-1

	Grafo *g = caminho(n);
	inserirAresta(g, 0, n-1);	
	return g;
}

#endif // FABRICA_H_INCLUDED\end{lstlisting}

\end{document}