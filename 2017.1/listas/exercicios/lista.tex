\documentclass[12pt,a4paper,oneside]{article}

\usepackage[utf8]{inputenc}
\usepackage[portuguese]{babel}
\usepackage[T1]{fontenc}
\usepackage{amsmath}
\usepackage{amsfonts}
\usepackage{amssymb}
\usepackage{hyperref}
\usepackage{url}

\usepackage{xcolor}
% Definindo novas cores
\definecolor{verde}{rgb}{0.25,0.5,0.35}
\definecolor{jpurple}{rgb}{0.5,0,0.35}

\author{\\Universidade Federal de Goiás (UFG) - Regional Jataí\\Bacharelado em Ciência da Computação \\Teoria de Grafos - 2017.1 \\Prof. Esdras Lins Bispo Jr.}
\date{}

\title{
	\sc \huge Listas de Exercícios 
	\\{\tt Versão 4.5}
}


\begin{document}

\maketitle

\section{Livro de Referência}
	\begin{itemize}
		\item FEOFILOFF, P. {\bf Exercícios de Teoria dos Grafos}, BCC, IME-USP, 2012, \href{https://github.com/bispojr/grafos/raw/master/2017.1/arquivos/exercicios-grafos.pdf}{<<link do arquivo no repositório>>}.
	\end{itemize}
	
\section{Listas de Exercícios}

\begin{enumerate}

	\subsection{Teste 1}
	\item[] {\bf Grafos:} E 1.1, E 1.2, E 1.3, E 1.4, E 1.22;
	\item[] {\bf Vizinhanças e Graus de Vértices:} E 1.33, E 1.42, E 1.46;
	\subsection{Teste 2}
	\item[] {\bf Grafos Bipartidos:} E 1.25, E 1.26, E 1.27, E 1.29;
	\item[] {\bf Caminhos e Circuitos:} E 1.57,	E 1. 64, E 1.65,	E 1.68;
%	\item[] {\bf Cortes:} E 1.103;
	\subsection{Teste 3}
	\item[] {\bf Caminhos e circuitos em grafos:} E 1.116,	E 1.117, E 1.124;
	\item[] {\bf Grafos Conexos:} E 1.140, E 1.42, E 1.144;
	\item[] {\bf Componentes:} E 1.171, E 1.172;
	
	\subsection{Teste 4}
	\item[] {\bf Pontes:} E 1.194, E 1.196;
	\item[] {\bf Florestas e Árvores:} E 1.222, E 1.223;
	\item[] {\bf Isomorfismo:} E 2.1, E 2.3;
	\item[] {\bf Questões de implementação:} Em linguagem C, utilizando as bibliotecas criadas na disciplina, escreva as funções abaixo:
	\begin{enumerate}
		\item {\tt int ehArvore(Grafo *g)}: o objetivo é que esta função retorne valor 1, se o grafo fornecido como parâmetro for uma árvore. O valor de retorno será 0, caso não for.
		\item {\tt int ehPonte(Grafo *g, int v, int w)}: o objetivo é que esta função retorne o valor 1, se a aresta {\tt vw} for ponte no grafo fornecido. O valor de retorno será 0, caso não for.
		\item {\tt int contemCircuito(Grafo *g)}: o objetivo é que esta função retorne o valor 1, se grafo fornecido como parâmetro contém algum circuito. O valor de retorno será 0, caso não for.
	\end{enumerate}
    
%	\item[] {\bf Grafos Dirigidos e Algoritmos em Grafos:} Q 01.

\end{enumerate}

\end{document}