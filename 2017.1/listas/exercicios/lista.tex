\documentclass[12pt,a4paper,oneside]{article}

\usepackage[utf8]{inputenc}
\usepackage[portuguese]{babel}
\usepackage[T1]{fontenc}
\usepackage{amsmath}
\usepackage{amsfonts}
\usepackage{amssymb}
\usepackage{hyperref}
\usepackage{url}

\usepackage{xcolor}
% Definindo novas cores
\definecolor{verde}{rgb}{0.25,0.5,0.35}
\definecolor{jpurple}{rgb}{0.5,0,0.35}

\author{\\Universidade Federal de Goiás (UFG) - Regional Jataí\\Bacharelado em Ciência da Computação \\Teoria de Grafos - 2017.1 \\Prof. Esdras Lins Bispo Jr.}
\date{}

\title{
	\sc \huge Listas de Exercícios 
	\\{\tt Versão 1.0}
}


\begin{document}

\maketitle

\section{Livro de Referência}
	\begin{itemize}
		\item FEOFILOFF, P. {\bf Exercícios de Teoria dos Grafos}, BCC, IME-USP, 2012, \href{https://github.com/bispojr/grafos/raw/master/2017.1/arquivos/exercicios-grafos.pdf}{<<link do arquivo no repositório>>}.
	\end{itemize}
	
\section{Listas de Exercícios}

\begin{enumerate}

	\item[] {\bf Grafos:} E 1.1, E 1.2, E 1.3, E 1.4, E 1.22;
%	\item[] {\bf Grafos Bipartidos:} E 1.25, E 1.26, E 1.27;
%	\item[] {\bf Vizinhanças e Graus de Vértices:} E 1.33, E 1.42, E 1.46;
%	\item[] {\bf Caminhos e Circuitos:} E 1.57,	E 1.65,	E 1.68;
%	\item[] {\bf Cortes:} E 1.103;
%	\item[] {\bf Caminhos e circuitos em grafos:} E 1.116,	E 1.117;
%	\item[] {\bf Grafos Conexos:} E 1.140;
%	\item[] {\bf Componentes:} E 1.171;
%	\item[] {\bf Pontes:} E 1.194;
%	\item[] {\bf Florestas e Árvores:} E 1.222;
%	\item[] {\bf Isomorfismo:} E 2.1;
%	\item[] {\bf Grafos Dirigidos e Algoritmos em Grafos:} Q 01.

\end{enumerate}

\end{document}