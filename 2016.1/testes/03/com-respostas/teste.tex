\documentclass[12pt,a4paper,oneside]{article}

\usepackage[utf8]{inputenc}
\usepackage[portuguese]{babel}
\usepackage[T1]{fontenc}
\usepackage{amsmath}
\usepackage{amsfonts}
\usepackage{amssymb}
\usepackage{graphicx}

\usepackage{xcolor}
% Definindo novas cores
\definecolor{verde}{rgb}{0.25,0.5,0.35}
\definecolor{jpurple}{rgb}{0.5,0,0.35}
% Configurando layout para mostrar codigos Java
\usepackage{listings}
\lstset{
  language=Java,
  basicstyle=\ttfamily\small, 
  keywordstyle=\color{jpurple}\bfseries,
  stringstyle=\color{red},
  commentstyle=\color{verde},
  morecomment=[s][\color{blue}]{/**}{*/},
  extendedchars=true, 
  showspaces=false, 
  showstringspaces=false, 
  numbers=left,
  numberstyle=\tiny,
  breaklines=true, 
  backgroundcolor=\color{cyan!10}, 
  breakautoindent=true, 
  captionpos=b,
  xleftmargin=0pt,
  tabsize=4,
  escapeinside=||
}

\author{\\Universidade Federal de Goiás (UFG) - Regional Jataí\\Bacharelado em Ciência da Computação \\Teoria dos Grafos \\Esdras Lins Bispo Jr.}

\title{\sc \huge Terceiro Teste}

\date{28 de junho de 2016}

\begin{document}

\maketitle

{\bf ORIENTAÇÕES PARA A RESOLUÇÃO}

\footnotesize

\begin{itemize}
	\item A avaliação é individual, sem consulta;
	\item A pontuação máxima desta avaliação é 10,0 (dez) pontos, sendo uma das 05 (cinco) componentes que formarão a média final da disciplina: dois testes, duas provas e exercícios;
	\item A média final ($MF$) será calculada assim como se segue
	\begin{eqnarray}
		MF & = & MIN(10, S) \nonumber \\
		S & = & (\sum_{i=1}^{4} 0,2.T_i ) + 0,2.P  + 0,1.E \nonumber
	\end{eqnarray}
	em que 
	\begin{itemize}
		\item $S$ é o somatório da pontuação de todas as avaliações,
		\item $T_i$ é a pontuação obtida no teste $i$,
		\item $P$ é a pontuação obtida na prova, e
		\item $E$ é a pontuação total dos exercícios.
	\end{itemize}
	\item O somatório da pontuação de todas as questões desta avaliação é 11,0 (onze) pontos. Isto é um sinônimo de tolerância na correção. Se você por acaso perder 1,5 (um e meio), sua nota será 9,5 (nove e meio);
	\item O conteúdo exigido compreende os seguintes pontos apresentados no Plano de Ensino da disciplina: (5) Cortes e Pontes, (6) Árvores, (9) Planaridade e (10) Outros tópicos.
\end{itemize}

\begin{center}
	\fbox{\large Nome: \hspace{10cm}}
	\fbox{\large Assinatura: \hspace{9cm}}
\end{center}

\newpage

\normalsize

\begin{enumerate}

	\item (5,0 pt) [E 1.110] Responda, justificando a sua resposta:
		\begin{enumerate}
			\item Suponha que todos os vértices de um grafo $G$ têm grau par. É verdade que d($X$) é par para todo subconjunto $X$ de $V_G$?
			
	\vspace{0.3cm}
	
	{\color{verde}
		R - Sim, é verdade. Será provado por indução que d($X$) é par para todo subconjunto $X$ de $V_G$, se todos os vértices de um grafo $G$ têm grau par. A indução será feita em $|X|$.
		\begin{itemize}
			\item {\bf Caso básico 1:} $|X| = 0 \therefore$ Como $\nabla(X) = \emptyset$, então d($X$) = 0 e 0 é par.
			\item {\bf Caso básico 2:} $|X| = 1 \therefore$ Seja $v$ o único vértice de $X$. Logo $|\nabla(X)|$ será par, pois todos os vértices de $G$ têm grau par. Assim temos que d($X$) é par, pois d($X$) = d($v$).
			\item {\bf Caso Geral:} $|X| > 1 \therefore$ 
				\begin{itemize}
					\item {\bf Hipótese de Indução:} d($X$) é par, para $|X| = k$;
					\item {\bf Passo da Indução:} d($X'$) é par, para $|X'| = k + 1$ $\therefore$ \\
					Seja $y$ o vértice que pertence a $X'$ e não pertence a $X$. Podemos calcular d($X'$) desta forma:
					\begin{center}
						d($X'$) = d($X$) - $d_X(y)$ + $d_{X'}(y)$
					\end{center}
					em que 
					\begin{itemize}
						\item $d_X(y)$ é a quantidade de vizinhos de $y$ que estão em $X$, e
						\item $d_{X'}(y)$ é a quantidade de vizinhos de $y$ que estão em $X'$.
					\end{itemize}  
					Ora só há duas possibilidades, tendo em vista que $d(y)$ é par:
					\begin{enumerate}
						\item \underline{$d_{X}(y)$ e $d_{X'}(y)$ são pares}: d($X'$) é par, pois 
						\begin{center}
							{\sc Par} - {\sc Par} + {\sc Par} = {\sc Par}
						\end{center} 
						\item \underline{$d_{X}(y)$ e $d_{X'}(y)$ são ímpares}: d($X'$) é par, pois 
						\begin{center}
							{\sc Par} - {\sc Ímpar} + {\sc Ímpar} = {\sc Par}
						\end{center} 
					\end{enumerate}
				\end{itemize}
		\end{itemize}
		Logo, provamos que d($X$) é par para todo subconjunto $X$ de $V_G$, se todos os vértices de um grafo $G$ têm grau par $\blacksquare$
		}
		
		\newpage
		
			\item Suponha que todos os vértices de um grafo $G$ têm grau ímpar. É verdade que d($X$) é ímpar para todo subconjunto próprio e não vazio $X$ de $V_G$?
					
	\vspace{0.3cm}
	
	{\color{verde}
		R - Não é verdade. Será feita uma prova por absurdo. Suponha que seja verdade a afirmação dada. Seja $G = (V,A)$ um grafo de forma que $V=\{a,b,c,d\}$ e $A=\{ab, cd\}$. Se admitirmos $X = \{ a, c \}$, temos que $X$ é subconjunto próprio e não vazio de $V_G$. Deste modo, d($X$) = 2 e 2 é par (o que é um absurdo). Logo, não é verdade que d($X$) é ímpar para todo subconjunto próprio e não vazio $X$ de $V_G$ $\blacksquare$
	} 
		\end{enumerate}
		
	\newpage
	
	\item (5,0 pt) [E 1.226] Mostre que um grafo é uma floresta se e somente se tem a seguinte propriedade: para todo par $(x, y)$ de seus vértices, existe no máximo um caminho com extremos $x$ e $y$ no grafo.
	
	\vspace{0.3cm}
	
	{\color{verde}
		R - A prova será dividida em duas partes: 
		\begin{enumerate}
			\item se em um grafo todo par $(x, y)$ de seus vértices tiver no máximo um caminho com extremos $x$ e $y$, então este grafo é uma floresta;
			\item se um grafo for uma floresta, então, neste grafo, todo par $(x, y)$ de seus vértices tem no máximo um caminho com extremos $x$ e $y$.
		\end{enumerate}
		
		\noindent\rule{12cm}{0.4pt}
		
		{\bf Parte (a):} Ora, se todo par $(x, y)$ de vértices de um grafo tiver no máximo um caminho com extremos $x$ e $y$, então este grafo não tem circuitos. Isto é verdade, pois se há circuitos em um grafo, pelo menos um par de vértice tem dois caminhos distintos que os ligam. Se este grafo não tem circuitos, então este grafo é uma floresta.
		
		\noindent\rule{12cm}{0.4pt}
		
		{\bf Parte (b):} Se um grafo for uma floresta, então este grafo só tem pontes. Logo, para todo par $(x, y)$ de vértices deste grafo pode haver:
		\begin{itemize}
			\item Nenhum caminho que liga $x$ a $y$, \\se $x$ e $y$ estiverem em componentes distintos;
			\item Um caminho, se $x$ e $y$ estiverem no mesmo componente (pois todo componente é conexo).
		\end{itemize}
		Pode-se garantir que não há mais de um caminho com extremos $x$ e $y$, pois, se houvesse, então este grafo teria um circuito contendo os vértices $x$ e $y$. Mas isto é impossível, pois este grafo só tem pontes e, consequentemente, não tem circuitos. Logo, neste grafo, todo par $(x, y)$ de seus vértices tem no máximo um caminho com extremos $x$ e $y$.
		
		\noindent\rule{12cm}{0.4pt}		
		
		Como (a) e (b) são verdadeiras, logo um grafo é uma floresta se e somente se tem a seguinte propriedade: para todo par $(x, y)$ de seus vértices, existe no máximo um caminho com extremos $x$ e $y$ no grafo $\blacksquare$
	}
	
	\end{enumerate}
\end{document}