\documentclass[12pt,a4paper,oneside]{article}

\usepackage[utf8]{inputenc}
\usepackage[portuguese]{babel}
\usepackage[T1]{fontenc}
\usepackage{amsmath}
\usepackage{amsfonts}
\usepackage{amssymb}
\usepackage{graphicx}

\usepackage{xcolor}
% Definindo novas cores
\definecolor{verde}{rgb}{0.25,0.5,0.35}
\definecolor{jpurple}{rgb}{0.5,0,0.35}
% Configurando layout para mostrar codigos Java
\usepackage{listings}
\lstset{
  language=Java,
  basicstyle=\ttfamily\small, 
  keywordstyle=\color{jpurple}\bfseries,
  stringstyle=\color{red},
  commentstyle=\color{verde},
  morecomment=[s][\color{blue}]{/**}{*/},
  extendedchars=true, 
  showspaces=false, 
  showstringspaces=false, 
  numbers=left,
  numberstyle=\tiny,
  breaklines=true, 
  backgroundcolor=\color{cyan!10}, 
  breakautoindent=true, 
  captionpos=b,
  xleftmargin=0pt,
  tabsize=4,
  escapeinside=||
}

\author{\\Universidade Federal de Goiás (UFG) - Regional Jataí\\Bacharelado em Ciência da Computação \\Teoria dos Grafos \\Esdras Lins Bispo Jr.}

\title{\sc \huge Prova (Parte 1)}

\date{16 de agosto de 2016}

\begin{document}

\maketitle

{\bf ORIENTAÇÕES PARA A RESOLUÇÃO}

\footnotesize

\begin{itemize}
	\item A avaliação é individual, sem consulta;
	\item A pontuação máxima desta avaliação é 10,0 (dez) pontos, sendo uma das 05 (cinco) componentes que formarão a média final da disciplina: dois testes, duas provas e exercícios;
	\item A média final ($MF$) será calculada assim como se segue
	\begin{eqnarray}
		MF & = & MIN(10, S) \nonumber \\
		S & = & (\sum_{i=1}^{4} 0,2.T_i ) + 0,2.P  + EB \nonumber
	\end{eqnarray}
	em que 
	\begin{itemize}
		\item $S$ é o somatório da pontuação de todas as avaliações,
		\item $T_i$ é a pontuação obtida no teste $i$,
		\item $P$ é a pontuação obtida na prova, e
		\item $EB$ é a pontuação total dos exercícios-bônus.
	\end{itemize}
	\item O conteúdo exigido compreende os seguintes pontos apresentados no Plano de Ensino da disciplina: (1) Noções Básicas de Grafos, (2) Caminhos e Circuitos, (3) Subgrafos e (4) Grafos conexos e componentes.
\end{itemize}

\begin{center}
	\fbox{\large Nome: \hspace{10cm}}
	\fbox{\large Assinatura: \hspace{9cm}}
\end{center}

\newpage

\normalsize

\begin{enumerate}

	\section*{Primeiro Teste}

	\item (5,0 pt) No vídeo do Prof. Paulo Cezar, é utilizado o conceito de circuito euleriano. Apresente todas as condições necessárias para um determinado grafo conexo conter um circuito euleriano.

	\vspace{0.3cm}	
	
	{\color{blue} {\bf R -} Um grafo conexo contém um circuito euleriano se todos os vértices deste grafo têm grau par. }
	
	\vspace{0.3cm}
	
	\item (5,0 pt) [E 1.68] É verdade que todo grafo 2-regular é um circuito? Justifique a sua resposta.
	
	\vspace{0.3cm}	
	
	{\color{blue} {\bf R -} Não é verdade. Podemos construir o grafo $G$, com dois componentes $C_1$ e $C_2$, de forma que $C_1$ seja um circuito de comprimento 4 e $C_2$ seja um $K_3$. $G$ é 2-regular, mas não é conexo. Logo não é um circuito. }
	
	\newpage
	
	\section*{Segundo Teste}

	\item (5,0 pt) [E 1.143] Sejam $P$ e $Q$ dois caminhos tais que $V_P \cap V_Q \not= \emptyset$. Mostre que o
grafo $P \cup Q$ é conexo.

	\vspace{0.3cm}	
	
	{\color{blue} {\bf R -} Se $V_P \cap V_Q \not= \emptyset$, então existe ao menos um vértice $x$ em comum aos dois caminhos. Logo, a partir de dois vértices quaisquer $v$ e $w$ de $P \cup Q$, temos:
		\begin{itemize}
			\item \underline{se $v, w \in V_P$:} 
			\\$v$ liga-se a $w$ por um caminho que é um subcaminho de $P$;
			\item \underline{se $v, w \in V_Q$:} 
			\\$v$ liga-se a $w$ por um caminho que é um subcaminho de $Q$;
			\item \underline{se $v \in V_P$ e $w \in V_Q$:} 
			\\$v$ liga-se a $x$ por um caminho que é um subcaminho de $P$, e 
			\\$x$ liga-se a $w$ por um caminho que é um subcaminho de $Q$; 
			\\logo $v$ liga-se a $w$;
			\item \underline{se $v \in V_Q$ e $w \in V_P$:} 
			\\$v$ liga-se a $x$ por um caminho que é um subcaminho de $Q$, e 
			\\$x$ liga-se a $w$ por um caminho que é um subcaminho de $P$; 
			\\logo $v$ liga-se a $w$;
		\end{itemize}
	Como sempre existe um caminho que liga $v$ a $w$, então $P \cup Q$ é conexo $\blacksquare$
	
	}
	
	\newpage

	\item (5,0 pt) [E 1.177] Seja $G$ um grafo tal que $\Delta(G) \leq 2$. Descreva os componentes de $G$.
	
	\vspace{0.3cm}	
	
	{\color{blue} {\bf R -} Podemos listar os três possíveis casos em relação ao valor de $\Delta(G)$:
		\begin{itemize}
			\item \underline{$\Delta(G) = 0$:} 
			\\neste caso, cada componente de $G$ é um $K_1$; 
			\\e $G$ tem $n$ componentes;
			\item \underline{$\Delta(G) = 1$:} 
			\\neste caso, cada componente de $G$ ou é um $K_1$, ou é um $K_2$; 
			\\e $G$ tem no mínimo $\lceil n/2 \rceil$ componentes e, 
			\\no máximo, $n -1$ componentes;
			\item \underline{$\Delta(G) = 2$:} 
			\\neste caso, cada componente de $G$ é 
				\begin{itemize}
					\item um $K_1$,
					\item um caminho, ou
					\item um circuito;
				\end{itemize}
			e $G$ tem no mínimo um componente e, 
			\\no máximo, $n - 2$ componentes.
		\end{itemize}
	
	}
	
	\end{enumerate}
\end{document}